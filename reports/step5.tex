\documentclass[UTF8]{ctexart}
\usepackage{graphicx}
\usepackage{amssymb}
\usepackage{amsmath}
\usepackage{subfigure}
\usepackage{geometry}
\usepackage{caption}
\usepackage{diagbox}
\usepackage{bm}
\usepackage{color}

\usepackage{enumitem}
\setitemize[1]{itemsep=0pt, partopsep=0pt, parsep=\parskip, topsep=5pt}

\usepackage{listings}
\lstdefinestyle{lfonts}{
    basicstyle = \ttfamily
}

\newcommand{\T}[1]{\texttt{{#1}}}
\title{Step 5 Report}
\author{刘轩奇 2018011025}
\date{2020年10月6日}
\geometry{left=2.0cm, right=2.0cm, top=2.5cm, bottom=2.5cm}
\begin{document}
    \maketitle
    \section{工作内容}
        本人选择不使用辅助工具 ANTLR 因此自己实现了 lexer 和 parser。
        \subsection{文件说明} 
            \T{montLexer.h/cpp} 词法分析器;

            \T{montParser.h/cpp} 语法分析器;

            \T{montConceiver.h/cpp} 产生中间代码;

            \T{montAssembler.h/cpp} 从中间代码产生汇编代码;
            
            \T{montLog.h} 记录编译错误信息;

            \T{montCompiler.cpp} MiniDecaf 编译器,包含主函数,编译成功返回 0 否则返回 -1,
            并将错误信息输出到 \T{std::cerr}。
        
        \subsection{本步骤完成的工作}

            \paragraph{1 词法分析} 添加了对注释的支持。遇到双斜线或斜线加星号时,忽略之后所有的注释。

            \paragraph{2 句法分析} 添加了以下新的 AST 节点,对应于指导书给出的非终结符:
            \begin{itemize}
                \item[*] \T{assignment}
                \item[*] \T{declaration} 
            \end{itemize}

            对 \T{statement} 的处理与指导书上有所不同,分为五种,即return语句、变量定义语句、表达式语句、
            代码块块语句(对应step7)、空语句。

            同时在 parse 的过程中统计每个节点的子树中所含有的变量声明占用空间的大小,以便后续处理。

            \paragraph{3 产生中间代码} 添加及修改了五种中间代码:
            \begin{itemize}
                \item[*] \T{BUILDFRAME} 在每个函数起始处使用,建立栈帧。带有一个整数参数为除了fp和ra以外所需的栈帧大小。
                \item[*] \T{FRAMEADDR, POP, LOAD, STORE} 与指导书上相同。
                \item[*] \T{RET} 作用与指导书上相同,但并不用跳转到一段特殊代码,而是直接弹出栈桢(\T{fp}的值给\T{sp})并
                据此还原原先的\T{fp}和\T{ra}(分别位于\T{sp}下4位和8位)。生成的具体代码详见下一节。
                。当遍历AST遇到函数末尾,而当前最后一条指令并非 \T{ret} 时
                ,自动生成 \T{PUSH 0} 和 \T{RET} 语句。
                \item[*] \T{LABEL} 标记函数起始处或跳转标记,在 step6 中会使用到。
            \end{itemize}

            在此步骤中同时检查变量的声明是否重复以及使用变量是是否已经定义。

            \paragraph{4 产生汇编代码} 新的或修改的中间代码将生成对应的汇编代码:
            \begin{itemize}
                \item[*] \T{[BUILDFRAME]} \T{sw ra, -4(sp); sw fp, -8(sp); ori fp, sp, 0; addi sp, sp, -8+4*k;}
                \item[*] \T{[FRAMEADDR]} \T{addi sp, sp -4; addi t1, fp, -12-4*k; sw t1, 0(sp);}
                \item[*] \T{[LOAD]} \T{lw t1, 0(sp); lw t1, 0(t1); sw t1, 0(sp);}
                \item[*] \T{[STORE]} \T{lw t1, 4(sp); lw t2, 0(sp); sw t1, 0(t2); addi sp, sp, 4;} 
                \item[*] \T{[POP]} \T{addi sp, sp, 4;} 
                \item[*] \T{[RET]} \T{lw a0, 0(sp); addi sp, sp, 4; ori sp, fp, 0; } 
                \T{lw ra, -4(sp); lw fp, -8(sp); jr ra;} 
                \item[*] \T{[LABEL]} \T{LABEL\_NAME: }
            \end{itemize}
        
    \section{思考题}
        \paragraph{1} 描述程序运行过程中函数栈帧的构成,分成哪几个部分?每个部分所用空间最少是多少?
        \paragraph{答} 程序函数栈帧由保存的寄存器、局部变量和运算栈三部分构成。保存的寄存器至少占用8字节
        因为至少需要保存\T{fp}和\T{ra}寄存器;局部变量可以为空,则不需要占用空间;运算栈也可以为空,也不需要占用空间。
        \paragraph{2} 有些语言允许在同一个作用域中多次定义同名的变量,例如这是一段合法的 Rust 代码(你不需要精确了解它的含义,大致理解即可):
        
        \noindent\rule{\textwidth}{1pt}
        \begin{lstlisting}[style=lfonts]
fn main() {
  let a = 0;
  let a = f(a);
  let a = g(a);
}
        \end{lstlisting}
        \noindent\rule{\textwidth}{1pt}
        
        其中\T{f(a)}中的a是上一行的\T{let a = 0;}定义的,\T{g(a)}中的\T{a}
        是上一行的\T{let a = f(a);}。
        如果 MiniDecaf 也允许多次定义同名变量,并规定新的定义会覆盖之前的同名定义,请问在你的实现中,需要对定义变量和查找变量的逻辑做怎样的修改?
        \paragraph{答} 只需要将重复的定义视为赋值语句即可。
\end{document}
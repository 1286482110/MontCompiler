\documentclass[UTF8]{ctexart}
\usepackage{graphicx}
\usepackage{amssymb}
\usepackage{amsmath}
\usepackage{subfigure}
\usepackage{geometry}
\usepackage{caption}
\usepackage{diagbox}
\usepackage{bm}
\usepackage{color}

\usepackage{enumitem}
\setitemize[1]{itemsep=0pt, partopsep=0pt, parsep=\parskip, topsep=5pt}

\usepackage{listings}
\lstdefinestyle{lfonts}{
    basicstyle = \ttfamily
}

\newcommand{\T}[1]{\texttt{{#1}}}
\title{Step 9 Report}
\author{刘轩奇 2018011025}
\date{2020年10月7日}
\geometry{left=2.0cm, right=2.0cm, top=2.5cm, bottom=2.5cm}
\begin{document}
    \maketitle
    \section{工作内容}
        本人选择不使用辅助工具 ANTLR 因此自己实现了 lexer 和 parser。
        \subsection{文件说明} 
            \T{montLexer.h/cpp} 词法分析器;

            \T{montParser.h/cpp} 语法分析器;

            \T{montConceiver.h/cpp} 产生中间代码;

            \T{montAssembler.h/cpp} 从中间代码产生汇编代码;
            
            \T{montLog.h} 记录编译错误信息;

            \T{montCompiler.cpp} MiniDecaf 编译器,包含主函数,编译成功返回 0 否则返回 -1,
            并将错误信息输出到 \T{std::cerr}。
        
        \subsection{本步骤完成的工作}

            \paragraph{1 词法分析} 添加了新的 Token:
            \begin{itemize}
                \item[*] \T{BOOL} 关键字 bool;
                \item[*] \T{VOID} 关键字 void;
                \item[*] \T{COMMA} 逗号。
                \item[*] \T{TRUE} 关键字 true;
                \item[*] \T{FALSE} 关键字 false。   
            \end{itemize}

            \paragraph{2 句法分析} 添加或修改了以下的 AST 节点:
            \begin{itemize}
                \item[*] \T{parameters} 等同于实验指导书上的 \T{parameter\_list}。
                \item[*] \T{exprlist} 等同于实验指导书上的 \T{expression\_list}。
                \item[*] \T{unary, postfix} 同实验指导书。
                \item[*] \T{function} 函数节点,支持 \T{void, int, bool, char} 返回值,支持
                    \T{int, bool, char} 参数类型。
            \end{itemize}

            \paragraph{3 产生中间代码} 在此阶段,遍历语法树过程中也判断每一个树节点的数据类型,
            例如整数数值节点的数据类型为整数,整数加法节点的类型为整数,逻辑运算节点的类型为布尔类型,等等。
            据此得到的结果来判断函数调用等是否符合语法。另外也添加了类型转换(主要是整数转换为布尔类型)的支持。
            
            添加了中间代码 \T{CALL} 并修改了 \T{RET} 使之对应于函数调用。
            其中 \T{CALL} 包含一个字符串参数表示调用函数名,一个数字参数表示函数参数个数。另外添加了
            中间代码 \T{CALLV, RETV} 与 \T{void} 类型函数相对应。

            \paragraph{4 产生汇编代码} 同指导书上说明,不再赘述。
        
    \section{思考题}
        \paragraph{1} MiniDecaf 的函数调用时参数求值的顺序是未定义行为。试写出一段 MiniDecaf 代码,使得不同的参数求值顺序会导致不同的返回结果。
        \paragraph{答} 

        \noindent\rule{\textwidth}{1pt}
        \begin{lstlisting}[style=lfonts]

int add(int a, int b){
    return a+b;
}

int main(){
    int a = 1;
    return add(a=a+1, a=a*2);
}

        \end{lstlisting}
        \noindent\rule{\textwidth}{1pt}
\end{document}